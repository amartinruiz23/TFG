% !TeX root = ../libro.tex
% !TeX encoding = utf8
%
%*******************************************************
% Introducción
%*******************************************************

% \manualmark
% \markboth{\textsc{Introducción}}{\textsc{Introducción}}

\chapter{Resumen}

En este trabajo se estudia la aplicación de técnicas del ámbito del aprendizaje profundo en la generación de música. Se enuncia y demuestra el teorema de aproximación universal que caracteriza la capacidad representativa de las redes neuronales, se realiza un estudio de los fundamentos del aprendizaje profundo, las técnicas de aprendizaje profundo para la modelización de datos secuenciales y el aprendizaje de características. Se desarrollan los fundamentos matemáticos del \textit{autoencoder} variacional, basado en inferencia variacional, y se aplica en un modelo para la manipulación y generación de melodías musicales. Por último se desarrolla una herramienta como aplicación de dicho modelo.

Palabras clave: aprendizaje automático, aprendizaje profundo, inferencia variacional, autoencoder variacional, generación de música

\endinput
